\documentclass[a5paper,twoside,10.5pt]{ctexart}
\newcommand\specialsectioning{\setcounter{secnumdepth}{-2}}
\specialsectioning
\usepackage[center]{titlesec}
\usepackage{fancyhdr}
\pagestyle{fancy}%fancy style
\fancyhf{}%清空页眉页脚
\fancyhead[LE,RO]{\thepage}%页码位置:偶数页居左,奇数页居右
\fancyfoot[RO,RE]{\textit{Einleitung}}% 设置页脚:在每页的右下脚以斜体显示书名

\renewcommand{\headrulewidth}{0pt} % 页眉与正文之间的水平线粗细
\renewcommand{\footrulewidth}{0pt}

\usepackage{changepage}%设置引用段落左右侧缩进
\newCJKfontfamily\mincho{IPAexMincho}%日语明朝体

\title{政治经济学批判导论}
\author{马克思著 \quad 王苏谈译}

\begin{document}

\maketitle

\newpage

\tableofcontents

\newpage

\section{译者序言}

译者之翻译这篇经典文献,最初乃源于自己的研究对该德文文本有所需求。笔者长期以来研读马克思经济学,尤其关注其方法论问题。近来,为写作关于马克思《导论》的毕业论文,对《导论》展开了深入的文本分析。国内通行的一系列译本,皆由原中央编译局所译。但是,译者在查考原文时,逐渐对这系列的译本产生了严重不满。主要是因为它们暗中贯彻了一个意识形态的诉求,那就是把《导论》中的黑格尔-马克思关系尽可能捺除,以树立一个虚幻的理论,即“逻辑与历史的统一”。但是,译者的根本立场是:要想真正理解马克思的《导论》,必须彻底摈弃传统诠释体系的逻辑-历史的方式,而真正承认《导论》同黑格尔哲学、尤其《法哲学》的密切关联。

笔者遂从德文原文全文翻译了《导论》,并力图恢复被传统诠释体系所刻意抹杀的辩证范畴。在研究和翻译过程中,最终得出了一个关于辩证范畴的翻译原则:马克思《导论》中的辩证范畴,必须尽可能吸收中文世界已有的黑格尔哲学中相应的成熟汉译。

翻译工作最初依据《马克思主义者互联网文库》(Marxists’ Internet Archive,简称MIA)中收录的\textit{Einleitung [zur Kritik der Politischen Ökonomie]},后又校以狄茨出版社1974年版的\textit{Zur Kritik der Politischen Ökonomie(Erstes Heft)}\footnote{Karl Marx, Zur Kritik der Politischen Ökonomie(Erstes Heft), Berlin: Dietz Verlag, 1974. }附录中的\textit{Karl Marx: Einleitung [zur Kritik der Politischen Ökonomie]}。凭借中译本校对了全文,而在重点段落上还参校了英译本和日译本\footnote{\mincho カ一ル·マルクス『経済学批判要綱』、大月書店、1974。}。

译者不担保翻译质量。译者以外的其他人所遭受的由使用此译本导致的一切问题或损失等等,皆由那些使用者本人承担,而译者恕不承担任何责任。译本可能会不定时更新,敬请谅解。

限于精力,译者未添加额外的注释,读者可自行参阅相关注释。本译本边码以上述1974年德文版为准。另附马恩选集第2卷2012年版的首位加上了@的页码。

此译本的翻译者是王苏谈,他保有由编译工作所带来的、受中国著作权法等法律、法规和政策以及国际上的相关公约、法律等所保护的一切著作权利。此译本\textit{自由地}供个人来学习和参考。任何人可在学习和研究这类\textbf{非商业用途}上自由使用此译本的全体或部分,但应该注意保证全文或部分引文的完整性,请勿断章取义。未经译者授权,不得用于商业用途。

如对该译本有任何意见或建议,可联系译者,我的邮箱是AAAAAOK@163.com。

此电子文档系借助\LaTeX 编译而成。其版面采用了A5样式,是A4的一半。分左右页。

最后,附上MIA原有介绍的译文:

%页码参阅:Seitenzahlen verweisen auf: Karl Marx/Friedrich Engels - Werke, (Karl) Dietz Verlag, Berlin. Band 13, 7. Auflage 1971, unveränderter Nachdruck der 1. Auflage 1961, Berlin/DDR. S. 615-641. [卡尔·马克思和弗里德里希·恩格斯:《著作》,柏林:(卡尔)迪茨出版社,第13卷,1971年第7版,即柏林/民主德国1961年第1版的未经改动的再版,第615-641页。]

\begin{adjustwidth}{2em}{2em}
\qquad{\fangsong “导言”,由马克思写作于1857年八月底到九月中旬,为他计划中的庞大的经济学著作提供了“总的导言”[“普遍的导论”]的一个未完成的草稿,马克思已经在“导言”中提出了这部著作的轮廓。在其他研究工作的进程中,马克思多次改变他原初的计划,而这就形成了著作《政治经济学批判》和《资本论》。\par “导言”在1902年在马克思的草稿中被发现,并在1903年首次出版于《新时代》杂志。发表的文本以马克思的手稿为基础。马克思的缩写单词默认改为全名。} 
\end{adjustwidth}

%2019年09月01日

\newpage

\marginpar{227 @683}

\section{\uppercase\expandafter{\romannumeral1}. 生产,消费,分配,交换(流通)}

\subsection{1.生产}

a) 当前的对象首先是\textbf{物质生产}。

在社会中生产的诸个人,因而诸个人之社会地规定的生产,自然是起点。单个、孤立的猎人、渔夫,斯密和李嘉图所由开始者,属于18世纪鲁宾逊小说式的无想象力的幻想。这种幻想,绝非如文化史家所幻想的那样,单纯表明对过度文雅的回击,及向被误解的自然生活的回归。很难说像卢梭的社会契约[contrat social],即生来独立的主体借由契约而转到关系[Verhältnis]和联系[ Verbindung]之中,以此种自然主义为基础。这是假象,仅仅是大大小小鲁宾逊小说式的美学假象。它其实是“市民社会”的预兆,这种“市民社会”自16世纪以来就在酝酿,而在18世纪的凯歌行进[Riesenschritte]中走向它的成熟。在这种自由竞争之社会中,出现了诸个人,从自然枷锁之类中解放出来者,这种自然枷锁使个人在以往的历史时代中,成为某个特定的、狭隘的人类集群之附属。18世纪的先知,斯密和李嘉图还完全站在其肩膀之上者,把这种18世纪的个人\marginpar{@684}——一方面是封建的社会形式之解体的产物,另一方面是自16世纪以来的新兴生产力的产物——认作典范,其实存是过去的事情。\marginpar{228}不是作为一种历史的结果,而是作为历史之起点。由于作为合乎天性的个人,适应于这些先知关于人类天性之表象,[故而]并非一种历史地产生的东西,而是生来就确立的东西。这种错觉为迄今的每个新时代所特有。斯图亚特,在相当多的方面站在18世纪的对立面,并且身为贵族更多地站在历史的基础之上,避免了此种昏钝。

我们越深入地追溯历史,就越看出来,个人,因而也是生产的个人,是不独立的,隶属于某种更大的整体的:早先还是以完全自然的方式,处于家庭之中,处于扩为氏族的家庭之中;随后则处于源自诸氏族之对抗和融合的公社之中,处于公社的各种形式之中。只是在18世纪,在“市民社会”中,社会性的相互联系之各种形式,才阻挡着个人,作为单纯的手段,用于诸个人的私人目的之上者,作为外在的必然性。然而那个御时者[Epoche],即产生孤立的个人这种观点者,正是迄今最发达的社会性的(从那个观点来看就是一般性的)关系。人在最严格的意义上是一种\textbf{城邦动物}\footnote{ζωον πολιτκον,希腊语},不只是一种群居动物,还是这么一种动物,它只有在社会中才能自我孤立。外在于社会的孤立的个人的生产——这种稀罕事,出于偶然而流落荒野的文明人也许会碰到,他们内在地在动力上已经具有了社会力量——其荒谬无异于语言学习[Sprachentwicklung]而无\textbf{一起}生活、一起交谈的个人。以上所述无需赘言。这一细节本来完全无需提及,如果这些蠢话,对于18世纪的公众来说具有意义和理由的蠢话,没有被巴师夏、凯里和蒲鲁东等人,重新煞有其事地吸收到最现代的经济学之中。于蒲鲁东等人而言,当然乐意对某种经济关系的起源,他不知其历史形成者,据此历史\marginpar{229}哲学地加以阐发:他编造神话说,亚当或普罗米修斯对那个固定的、完满的理念感到满意,于是那种理念就被采用了云云。没有什么比这类胡扯的套话更枯燥无聊了\marginpar{@685}。

因此只要谈论生产,就总是谈论生产位于某个特定的、社会性的发展阶段之上者——社会性的个人之生产。因此看起来也许就是说,总而言之,说到生产,我们要么必须把历史的发展过程在其各个阶段上加以追踪,要么一开始就声明,我们要将生产同某个特定的历史时代相关联,因而例如同现代的布尔乔亚的[bürgerlichen]生产相关联,而这事实上就是我们真正的主题。可是生产的所有时代都具有某些共同的特征、共同的规定。\textbf{广义生产}[Produktion im allgemeinen]是一个抽象,然而是一个聪明的[verständige]抽象,如果它确实突出、固定了共同点,而我们因此免除了重复。与此同时,这个\textbf{一般},或者说经比较而分离出来的共同点,自身即是一个多重的结构物[vielfach Gegliedertes],在不同的诸规定中分别开来的东西。其中一些属于所有时代;另一些共属于一些时代。诸规定而被最现代的时代同最古老的时代所共有,是存在的。若不允许设想它们,就不会有任何生产;然而,如果最发达的语言具有那些规则和规定,它们同最不发达的语言所共有,那么将它们的发展标识出来的,恰恰是那区别,源自这些一般和共同点者。诸规定,适应于生产一般[Produktion überhaupt]者,恰好必须被分开,以超出统一——这种统一已经从这里产生了:主体,人类,与客体,自然,皆无二致——而本质的差别不被忘记。在这种忘记中比方说就有现代经济学家的全部才智,他们要证明现有社会关系的永恒与和谐。例如,任何生产都无法进行,\marginpar{230}若没有某种生产工具,即便这种工具还仅是手。任何生产都无法进行,若没有过去的、凝结的劳动,即便这种劳动还仅是技能、即在野人手中借反复练习而积累和集中的技能。此外资本也是生产工具,也是过去的、客体化的劳动。所以资本是一种一般的、永恒的自然关系;也就是说,如果我恰恰忽略了那个特殊,即正是让“生产工具”“凝结的劳动”成为资本的那个特殊。于是生产关系的全部历史,比如对凯里来说,就表现为由政府恶意促成的某种歪曲。

如果不存在广义生产,那么也就不存在一般的生产。生产总是某个\marginpar{@686}\textbf{特殊的}生产部门——诸如农业、畜牧业、工业等等——或者它是\textbf{总体}。然而政治经济学并非实用技术。生产之一般规定在某个既定社会阶段上同特殊的生产形式的关系,在别处阐述(之后)。

最后,生产也并非仅是特殊的。而是说它始终不过是一个特定的社会机体、一个社会性的主体,它在诸生产部门的一个或较丰富或较贫乏的总体之中是活动着的[tätig]。科学的阐述对真实的运动所具有的关系,同样还不属于这里。广义生产。特殊的生产部门。生产之总体。

经济学流行把一个一般的部分事先派遣出来——而它恰恰冠以标题“生产”而出现(参见约·斯·穆勒的例子)——在其中所有生产的一般条件得以论述。这个一般的部分包括或者据说应该包括:

1. 生产要想可能所不可或缺的条件。也即,因此实际上不外指出所有生产的本质的环节。它实际上却被还原为,正如所见,若干非常简单的规定,在乏味的同义反复中被说教;

2. \marginpar{231}或多或少生产推动生产的条件,正像例如亚当·斯密发展的或萧条的社会状态。这一点,被他当作梗概[Aperçu]而有其价值者,要提升到科学意义上,研究,关于在各个民族的发展中的生产率程度之周期的研究,就有必要。这一研究,处在主题的真正边界之外,只要它还保持原样,就要被安排在竞争、积累等等的发展之中。在一般的含义上,答案导向[hinauslaufen]那一般,即一个工业民族之占据其生产顶峰,是在它一般地占领其历史顶峰的时刻。其实\footnote{英语},一个民族的工业顶峰,正在于仍非利润而是获利是它的要紧之事。如此,则美国佬胜过英格兰人。不然就是说:例如特定的种类、设备、气候、自然环境,诸如海洋状况和土地肥沃程度等等,对生产来说,比其他的更合适。答案重又导向那同义反复:财富\marginpar{687}的因素在主观上、客观上在更高程度上被具备,财富在那些程度上就更容易被创造。

然而这还不是经济学家在这个一般的部分中真正所讨论的全部内容。生产据说宁可——参见例如穆勒——区别于分配等等,被描绘成局限于同历史相独立的永恒的自然规律之内。借此机会,完全在暗地里,资产阶级关系作为抽象社会的无可辩驳的自然规律,被私运进来。这就是所有操作的或多或少故意的目的。至于分配则与此相反,据说人们实际上被容许有各种各样的任性。完全撇开生产同分配的粗暴撕裂以及二者的实际关系,总该起初就明白,若干社会阶段上的分配无论怎么多样,都必定同样可以像在生产中那样,同样好地得到共同的规定,而且同样可以把所有历史性的区别,都混淆或抹杀于\textbf{一\marginpar{232}般的人类的}规律之中。例如奴隶、农奴、雇佣工人全都获得一定份额的饮食,以使他们能够作为奴隶、农奴、雇佣工人存活下去。仰赖贡金的征服者,仰赖税收的公职人员,仰赖地租的地产终身保有者,仰赖施舍的修道士,或仰赖什一税的祭司,都获得社会生产的一个商[quotum]。同奴隶之属的商相比,这种商是按照不同的规律来确定的。有两个要点,所有经济学家在这种标题下都要提出,就是:1. 所有[Eigentum];2. 由司法、警察等等提供的对所有的保障。对此非常简短地回应如下:

关于1. 一切生产都是对自然的占有,这种占有从个体方面来看,内在于且借助于某种特定的社会形式。在这个意义上,这样说就是同义反复:所有(占有[Aneignen])是生产的前提。然而荒谬的是,从这里就一步跨到了所有的某种特定形式,例如私人所有。(它仍是一种矛盾的形式,在同样程度上把\textbf{一无所有}[Nichteigentum]当作前提来隶属。)历史宁可说证明了公有[Gemeineigentum](例如就印度人、斯拉夫人、古凯尔特人等来说)才是本源的形式,这种形式还在公社所有的型态[Gestalt]下长期扮演显著的角色。至于财富[Reichtum]是在所有之这种还是那种形式[Form]中发展得更好,这个问题,在这里还完全无需论及。然而\marginpar{@688}说什么没有所有形式的地方,就会谈不上任何生产,因而也就谈不上任何社会,这是同义反复。是一个占有,又不占有什么,这是一个悖论。

关于2. 对获得物的保障等等。如果这些陈词滥调被归结为它们的真正内容,那么它们所能传达的,比它们的传道者所知道的还多。换句话说,每种生产形式都生成与之相适应的法律关系和政体等等。粗暴和概念缺失就存在于这种做法之中:把有机的\marginpar{233}归属一体者偶然地关联起来,引致于某种纯粹的反思关系之中。资产阶级经济学家只知道,用现代警察相比诸如用动武权[Faustrecht],能更好地生产。他们只是忘记了,动武权也是一种法,而强者的法还以别的形式继续存在于他们的“法治国家”之中。

每当与某一特定生产阶段相适应的社会状态刚刚出现或已然消亡,当然会出现生产的混乱,虽然具有不同的程度和影响。

总结:所有生产阶段所共有的规定是存在的,它们被头脑固定为一般;但是所有生产的所谓\textbf{一般条件},却不外是那些抽象环节,真正的历史性的生产阶段借之无法被理解。

\newpage

\subsection{2.生产同分配、交换、消费的一般关系}
%\addcontentsline{toc}{section}{2.生产同分配、交换、消费的一般关系}

在进入对生产的进一步分析之前,有必要把经济学家同生产相对\footnote{neben,兼有在旁边和相对之意,日语译作ならべる,更贴合原意。}而提出的若干项目纳入视野。

那肤浅的、明摆着的表象是:在生产中社会成员适应(创制、塑造)自然产品以人的需要;分配规定那种比例,在其中个人参与这些产品;交换供应他以特殊的产品,他要转换他借助分配所分摊的份额为这些产品;最后在消费中产品\footnote{手稿作“生产”。}变成享受的对象、个体的占有物。生产创制同需要相适应的对象;分配分派它们以社会规律;交换再次分派那已经分派之物\marginpar{@689}以个人需要;\marginpar{234}最终在消费中,产品退出这一社会性的运动,直接成为个人需要之对象和奴仆,而在享受中使个人得到满足。因此生产显现为起点,消费显现为终点,分配和交换显现为中间点[Mitte],这中间点甚至又是二重的,因而分配作为从社会出发的环节而被规定,而交换作为从个体出发的环节而被规定。在生产中人客体化,在消费\footnote{手稿作“人”[Person]。}中物主体化;在分配中,社会以一般的、支配性的诸规定[Bestimmungen]之形式,充当生产与消费二者之间的中介[Vermittlung];在交换中,生产与消费由诸个体之偶然的规定性[Bestimmtheit]来中介[vermittelt]。

分配规定比例(量),在其中产品归于个人;交换规定产品,在其中个体要求他依分配而分派的份额。

生产,分配,交换,消费,因此构成一个合规的推论:生产[即]一般性,分配和交换[即]特殊性,消费[即]单一性,整体[便]在此中联结起来。这固为一种关联[Zusammenhang],却是一种乏味的[关联]。生产由一般性的自然规律所规定;分配由社会性的偶然[所规定],因而它可以或多或少对生产起促进作用;交换位于二者之间,作为形式上的社会性的运动,而消费之闭合行为——消费不仅被理解为终极目标[Endziel],而且还被理解为终极目的[Endzweck]——究竟位于经济学之外,除非是说,它又对那起点起反作用,并且重启那完整的过程。

政治经济学家的对手,无论是内在于还是外在于其白令[Berings],都指责政治经济学家把属于一体之物给野蛮地撕裂了。他们要么\marginpar{235}同政治经济学家站在同等的基础上,要么等而下之。没有什么比这种指责更鄙俗了,说政治经济学家唯独紧盯生产,视之为目的本身[Selbstzweck]。说分配同样非常重要。这种指责恰建基于那经济的表象,即分配作为自主的、独立的领域,相对于[neben]生产而栖息。或是指责说\marginpar{@690}诸环节未在它们的统一中被把握。说得好像这种撕裂不是从现实渗入教科书的,而是反倒从教科书渗入现实的,好像此处所涉及的,是对概念的某种辩证的调解[Ausgleichung],而非对真实的关系的那种诠解[Auflösung]!

\subsubsection{a) [生产与消费]}
%\addcontentsline{toc}{subsection}{a) [生产与消费]}

生产直接也是消费。二重的消费,主观的和客观的:在生产中其能力得到发展的个人,也在生产的行为中付出、消耗这些能力,完全就像自然的生育是从生命力而来的一种消费。其二:生产手段的消费,被利用和磨损,并部分地(就像例如在燃烧中)再次被分解为一般的要素。同样地,原料的消费,这些原料并不保持在其自然的形态和性质之中,其宁可说被耗费尽。生产的行为本身因此在其所有环节中同样是一种消费的行为。然而经济学家承认这一点。与消费直接地同一的生产,与生产直接地同时发生的消费,它们被叫做生产的消费。来自生产和消费的这种同一,归结为斯宾诺莎的定理:确定就是否定。

然而生产的消费的这个规定恰恰只是为此被分配,为了那与生产同一的消费而从那真正的消费中分离,真正的消费宁可作为\marginpar{236}生产的破坏性的对立面来理解。因此我们考察真正的消费。

消费直接也是生产,就像在自然中元素和化学物质的消费是植物的生产。例如在食物、即消费的某种形式中,人类生产其合适的身体,这很清楚。然而这一点适用于每种其他的消费行为,在这样那样的行为中,人类就某种方面来看都在生产。消费的生产。可是,经济学则说,那与消费同一的生产是某种第二位的生产,源出于第一位的生产的产品的毁灭。在那第一位的生产中,是生产者在失去人性,在那第二\marginpar{691}位的生产中,是由生产者所创造的事物在赋予人性。因此这一消费的生产——虽然它是生产与消费之间的某种直接统一——完全不同于真正的生产。生产同消费并且消费同生产在其中相一致的那直接的统一,允许它们的直接的二重性存在。

因此生产直接是消费,消费直接是生产。每一种都直接是其对立面。同时在两者之间又有一种调解的运动发生。生产调解消费,即创造消费的材料,没有生产就会缺失对象。然而消费也调解生产,产品的消费才创造主体,产品对主体来说是产品。产品在消费中才获得最终的结束。一条铁路,若无车辆行驶在上面,因此没有被磨损,没有被消费,就仅仅是一条在可能性而非现实性上的铁路。没有生产就没有消费;然而同样地,没有消费就没有生产,就此而言生产就会如此的没有意义。消费双重地生产生产:

1. 仅仅在消费中产品才成为实际的[wirklich]产品。例如,一件衣服之成为实际的衣服,只有借助穿的\marginpar{237}行为;一幢房屋,没人居住,事实上就不是实际的房屋;因此产品,区别于纯粹的自然对象,证明了只有在消费中才成为产品。消费是既存的,在此情况下,它消解产品,它才是最后一击;于是产品是生产,不仅作为具体化的活动,还仅作为对活动着的主体来说的对象;

2. 消费创造新生产之需要,因此是生产的理念的、内在地驱动的基础,生产的前提。消费创造生产的欲望;它同样创造在活动着的生产中是目的的对象。如果这是清楚的,即生产外在地提供消费的对象,那么由此同样清楚的是,消费理念地确定生产的对象,作为内在的图样、需要、欲望和目的。它仍然在主观的形式上创造生产的对象。没有需要就没有生产。然而消费再生产需要。

与\marginpar{@692}此相应,从生产的方面来看:

1. 生产为生产[疑为消费之误]提供材料、对象。没有对象的消费不是消费;因此从这方面,生产就创造、生产消费。

2. 消费的生产所创造的,不仅是对象。它还给予消费以它的规定性,它的性质,它的终结。同样就像消费给予产品一个作为产品的终结,生产给予消费以终结。首先,对象不能是对象一般,而是某种特定的对象,其必须以某种特定的、由生产自身再次调和的行为而被消费。饥饿就是饥饿,然而由烧熟的、用刀享用的肉来满足,不同于由生肉并借助于手、指甲和牙齿来吞咽。不仅仅是消费的对\marginpar{238}象,还有消费的方式,因此借由生产被生产,不仅客观上,而且主观上。因此生产创造消费者。

3. 生产不仅为需要提供某种材料,而且它还为材料提供某种需要。如果消费从其早期的自然粗野性和直接性中脱出——而在上述情况中的停留甚至是某种粘着在自然粗野性中的生产的结果——那么消费自身就作为欲望被对象所调和[vermittelt]。消费据之感受到的需要,借助自己的感觉而被创造。艺术对象——每种其他产品都一样——创造懂得欣赏艺术的、能够享受美的公众。因此生产不仅仅为主体生产某个对象,还为对象生产某个主体。

因而生产生产消费1. 当创造消费的材料;2. 当规定消费的方式;3. 当(生产)首先从消费的作为对象而成熟的产品,当作需要在消费中产生。生产因而生产消费的对象、方式、欲望。同样,消费生产生产者的资质,当消费作为确定目的的需要进行刺激。

消费与生产间的同一因此表现为三重的:

1. 直接的同一:生产是消费;消费是生产。消费的生产。生产的消费。国民经济学家将二者称作生产的消费。然\marginpar{@693}而仍做了一个区别。第一种作为再生产出现;第二种作为生产的消费出现。所有关于第一种的研究都是关于生产的或非生产的劳动;所有关于第二种的研究都是关于生产的或非生产的消费。

2. 每一种都作为中介[Mittel]显现另一种;由它而调和;都作为它们的相互的依赖性被表达;一种运动,借此它们彼此被建立关系而\marginpar{239}显现为相互不可或缺,然而自身不过仍外在地保持着。生产为消费创造作为外在对象的材料;消费创造作为内在对象、作为生产之目的的需要。无生产就无消费;无消费就无生产。上述情况在经济学中以多种形式出现。

3. 生产并不仅是直接的消费,而消费也不仅是直接的生产;生产还是对于消费的中介,而消费还是对于生产的目的,也即,每种都提供另一种以其对象,生产外在地提供消费以对象,消费想象地提供生产以基础;而且(前面都是不仅)上述每一种不仅仅直接是另一种,不仅仅还调和另一种,而且每一种都创造二者,只要自我贯彻,就创造它者。消费仅仅贯彻生产的行为,只要产品作为完成的产品;只要它消解、消耗它的独立的实际的形式;只要在第一位的生产行为中发展的资质借由重复的需要而增加技能;因此它不仅仅是结算的行为,借此产品成为产品,而且同样,由此生产者成为生产者。另一方面生产生产消费,只要它创造消费的特定方式,并且此后,只要它创造消费的吸引力、消费能力自身,作为需要。在3个特定的同一之中的最后这一种,在经济学中反复解释,在需求与供给、对象与需要、社会所创造的需要与自然的需要这些关系中。

于是对于一个黑格尔派来说,再简单不过的就是把生产和消费确立为同一的。而且这不仅由社会主义的纯文学家、而且甚至由散文体(乏味)的经济学家所实现,例如萨伊,在下述形式上,即如果考察一个民族,\marginpar{@694}那么其生产就是其消费。要不然就是抽象的人类。Storch(斯多尔希)证实了萨伊是错的,只要一个民族例如并不完全消费它的产品,而是\marginpar{240}同样创造生产资料[Produktionsmittel]等等,固定资本等等。把社会当作某种唯一的主体来考察,是对它作了错误的——抽象的——考察。对某个主体来说,生产和消费表现为某个行为的环节。最重要的是在这里强调:无论把生产和消费当作是某个主体还是单个个人的活动,它都表现为某个过程的环节,在此过程中生产便是真正的起点以及为此同样是统摄的环节。作为必需、需要的消费甚至是生产性的活动的一个内在环节。然而最后的[即消费]却是实现的起点以及因此同样是实现的统摄的环节、全部过程自身在其中再次消解的行为。个人生产一个对象并借由它的消费再次复归于自己,然而是作为生产性的个人,并且再生产自己本身。消费因此表现为生产的环节。

在社会中诸生产者对产品的关系却是——一旦产品完成了——某种外在的,并且产品向主体的复归取决于产品同其他个人的关系。这些产品并非被他直接获得。对产品的直接的占有同样不是他的目的,如果他在社会中生产的话。生产者与产品间插入了分配。分配根据社会规律确定生产者在产品世界中的份额,因此插入于生产和消费之间。

那么分配作为自给自足的领域位于生产之旁、之外吗?

\subsubsection{b) [生产和分配]}
%\addcontentsline{toc}{subsection}{b) [生产和分配]}
如果考察通常的经济学,必定首先注意到,其中一切都被双重地确立。例如在分配中出现了地租、劳动工资、利息和利\marginpar{241}润,然而在生产中土壤、劳动、资\marginpar{@695}本作为生产的要素[Agenten]出现。就资本来说马上就很清楚,它是二重地确立了的,1. 作为生产要素;2. 作为收入来源;作为支配性的特殊的分配形式。利息和利润因此也如此出现在生产之中,就此而言它是形式,在这些形式中资本自我增殖、自我增加,因此是其生产本身的环节。作为分配形式的利息和利润隶属于作为生产之要素的资本。它们是分配方式,以作为生产要素的资本为前提。它们同样是资本的再生产方式。

劳动工资同样是在另一种范畴下考察的工资劳动:劳动在这里作为生产要素所具有的规定性表现为分配规定。若劳动不是规定为工资劳动,那么这行为就像它之参与产品那样,不是表现为劳动工资,就像例如在奴隶制中。最后,地租(um gleich,分配的最发达形式,在这里地产参与产品)隶属于大地产(实际上是大农业),是作为生产要素,而非土壤绝对[die Erde schlechthin],就如同报酬不是隶属于劳动[die Arbeit schlechthin]。分配关系、分配方式因此仅仅表现为生产要素的背面。以工资劳动之形式参与的某个个人,以劳动工资之形式参与产品即生产之结果。分配之划分完全由生产之划分来规定。分配甚至是生产的一种产品,不仅是根据对象,即生产之结果能被分配,而且同样根据形式,即在生产中的参与之特殊的方式规定分配之特殊的形式,是这种形式,在其中得以参与分配。彻底是幻觉的是,在生产中安排土壤,在分配上安排某种地租,等等。经济学家如李嘉图,最受责备的是,他仅看到生产,因此把分配规定为经济学对象,因为他本能地把分配形式召集为最确定的表现,在其中生产要素固定在某种既存的社会中。[teilnimmen,部分+拿来]

如李嘉图那样的诸经济学家,最为人诟病的就是,他们\marginpar{242}仅关注生产,[却]在这里专门把分配规定为经济学的对象,因为他们本能地把分配形式当作那种最明确的表现,在这种表现中,某个既定社会中的诸生产要素就确定下来。

在单个的个人面前,分配自然显现为一种社会的规律,这种规律决定他在生产之内的地位,他[又]是在这个地位之内进行生产的,因此分配位于生产\marginpar{@696}之前。这种个人,从最初以来,就没有资本,[也]没有地产。自出生以来,就由于那社会的分配,而依赖工资劳动。但是这种依赖本身,是[下述情况的]结果,即资本、地产作为独立的生产要素而存在。

考察整个社会的话,分配似乎还从一方面优先于生产,并且规定生产;好像是前经济的事实。一个征服民族将土地分派在那征服者之中,并且因此铭刻了地产的一种特定的分派和形式:因而规定了生产。或者它使那被征服的[民族]成为奴隶,并因而使奴隶劳动成为生产之基础。或者一个民族,借助革命,把大地产打碎成小块土地;因此,借助这种新的分配,生产就获得一种新的性质。或者立法把地产分派给某种家庭,或者把劳动分派[为]世袭特权,因而在财政上[kostenmässig]将之固定。在所有这些史已有之的情况中,似乎不是分配由生产所划定[gegliedert]、规定,反倒是生产由分配所划定、规定。20200112

分配,以最乏味的观点来看,表现为产品的分配,因此远远离开并似乎\footnote{quasi}独立于生产。但是,在作为产品的分配之前,分配先得是:1.生产工具的分配,2.上述关系的一种进一步的规定,\marginpar{243}即社会成员[Mitglieder]之分配、其在各类生产行为之中者。(诸个人之归属、其在特定生产关系之下者。)产品的分配显然只是那种分配之结果,它内在地算入生产过程本身,并规定生产之划分。生产而不被放它所内含的分配之上来考察,显然就是空洞的抽象。[借此深入分析马克思的“抽象”概念]同时,反过来,产品的分配自动就被赋予了,由那种最初构成生产之一环节的分配[来赋予]。李嘉图致力于把现代生产放在其特定的社会的划分中理解,而且是[把]生产[视为]当务之急的经济学家,[却]恰恰由此宣称,并非生产而是分配,[才]是现代经济学的真正主题。这[种做法]在这里再次导致了那些经济学家的陈词滥调,把生产解释成永恒真理,同时把历史驱逐到分配领域。20200113

这一规定生产本身的分配对生产占据着怎样的关系,显然\marginpar{@697}是个内在地属于生产本身的问题。如果是说,至少生产必须肇始于生产工具的某种分配,所以分配在此种意义上先于生产,构成其前提。那就这样作答:生产其实具有其条件和前提,是它们构成了生产的诸环节。在最初的开端上,它们可能表现为顺应自然的。由于生产本身的过程,它们就被转变,从顺应自然的[转变]到根据历史的,并且对于一个时代表现为生产的自然性的前提,对于另一个时代就会表现为生产的历史性的结果。在生产本身内部,它们被持续改变。例如,机组的应用不仅改变了生产工具的分配,而且改变了产品的分配。现代大地产本身既是现代商业和现代工业的结果,也是后者[即现代工业]在农业中的应用的结果。

上面\marginpar{244}提的这些问题,完全归结为下述讼争:一般历史上的诸关系怎样包含于生产之中,以及它[生产]同现实运动一般[geschichtlichen Bewegung überhaupt]之关系如何。此类问题显然归属于对生产本身的研讨和阐释。20200114

即令是在平庸形式中,提出上述问题,仍可简略推断。所有征服都能分成三类。征服民族使被征服民族屈服于他们[征服民族]特有的生产方式(诸如在本世纪英格兰人在爱尔兰、部分地在印度[所做的那样]);或者延续旧的[生产方式]而满足于贡金(例如突厥人和罗马人);或者出现某种相互作用,由之形成某种新的[生产方式]、综合的[生产方式](部分地在日耳曼人的征服中)。在所有情形中,生产方式,不管它是征服民族的、被征服民族的、还是源于两相融合的,都规定了新出现的分配。虽然这种分配对新的生产时期显现为前提,但它本身甚至又是生产的产物,这种生产不仅是广义[im allgemeinen]历史上的,还是狭义[bestimmten]历史上的。

蒙古人,根据诸如他们在俄罗斯的破坏,认为他们的生产,即适度的放牧,对于荒无人烟的辽阔地带来说,是个首要条件。在日耳曼蛮族而言,由农奴耕作是传统的生产,并且在乡村中不相往来。他们能\marginpar{@698}使罗马行省如此轻易地屈服于这些条件,因为那里发生的地产集中,已经完全推翻了旧的农业关系。

有个传统的表象是,在某些时代能单靠掠夺过活。然而为了能够掠夺,必须得有东西可供掠夺,因而[就需要]生产。而且掠夺的方式甚至又被生产的方式所规定。比方说,掠夺炒股民族并不似掠夺牧牛民族。20200115

\marginpar{245}在奴隶而言,是生产工具直接被掠夺。那么这些田地[landes]\footnote{,或译:国家、土地、地方等}的生产——为此奴隶才被掠夺——却必须被划分得容许奴隶劳动,或者(像在南美洲等地)必须要有某种适于奴隶的生产方式被创设出来。

法可使某种生产工具,例如土地,在某些家族中永续。这种法之具有经济意义,仅当大地产同社会生产相和谐之时,比方说在英格兰。在法兰西,小农业得到推动,而不顾大地产,因而大地产同样被革命打碎了。但是分割成小块土地[这个状态]之永续,能由法[保全]吗?财产不顾这种法,再次集中起来了。法对保持分配关系的影响,由此,法对生产的作用,要另加规定。20200115

\subsubsection{c) 交换最后以及流通}

流通自身仅仅是交换的一个特定环节,或者同样地是在总体上考察的交换。

交换仅仅是在生产和借之被规定的分配连同消费之间的一个调和的环节,后者却甚至表现为生产的一个环节,就上述情况而言,交换显然也作为环节算在后者之中。 

首先,明显的是,在生产本身中发生的关于活动和能力的交换,直接就归属并实质上构成生产。第二,只要产品的交换\marginpar{@699}对于制成直接的消费所规定的产品而言是手段,那么它也适用于这同种情况。就此而言,交换本身就是算入生产的行为。第三,商人之间的所谓交换,其组织还完全由生产所规定,作为\marginpar{246}自我生产的活动。交换表现为在生产之旁独立,相对生产而中立,仅仅是在这最后的状态中,即产品直接为了消费而被交换。然而,1. 没有分工就没有交换,而分工现在是天然的或者甚至已经是历史的结果;2. 私人交换以私人生产为前提;3. 交换的深度及其广度和方式,都由生产的发展和划分所规定。例如,乡村里、城市中以及二者之间的交换等等。交换就这样表现在生产的所有环节里,要么直接包括在其中,要么由之被规定。

我们所得到的结果,并不是说生产、分配、交换、消费是同一的,而是说它们最终[alle]构成一个总体的肢节[Glieder],一个统一之内的区别。生产不仅统摄[greift über]自身于生产的对立的规定之中,而且也统摄其他的诸环节。过程总是从生产重又从新开始。交换和消费不能是统摄者[Übergreifende],这不言自明。作为产品之分配的分配也是如此。然而作为生产要素[Produktionsagenten]之分配,分配自身是生产的一个环节。因此某种特定的生产规定特定的消费、分配、交换,\textbf{这些各种各样的环节相互间的特定关系}。固然,生产同样\textbf{在其单方面上},就这个方面而言被其他诸环节所规定。例如,如果市场也即交换领域自身扩展,那么生产就根据那个规模而扩大,并更加细分。伴随着分配的变动,生产就要变动;例如伴随着资本的集中,人口在城乡中的各种分布,等等[,生产就要变动]。最终,消费需求规定生产。各种各样的环节之间可以找到交互作用。这种情况,每个有机的整体都有。

\newpage

\subsection{3.政治经济学的方法}\marginpar{247@700}

〇如果我们政治经济学地考察某个既定国家,那么我们就开始于它的人口,这个人口在这些类别上的分布,[这些类别即是]城市、乡村、海洋,那各种生产部门,进出口,年度的生产和消费,商品价格等等。

〇显得正确的是,以实在者和具体者、以实际的前提来开始,从而比如在经济学中以人口、即全社会的生产行为的基础与主体来开始。然而,依照更仔细的考察,这种做法显示出是错误的。〇人口是一个抽象,如果我比如把构成人口的诸阶级略去。这些阶级又是一句空话,如果我不了解阶级所依据的诸因素,诸如工资劳动、资本等等。他们[这些因素]又隶属于交换、分工、价格等等。比如资本,没有工资劳动,没有价值、货币、价格等等,就什么都不是。〇因此,我以人口开始,那么人口就是整体的一个混乱表象,而经由更仔细的规定,我将分析地逐渐得出更简单的概念;从设想的具体者到愈发单薄的抽象者,直到我抵达那些最简单的规定。现在从那里又往回踏上旅程,直到我最终又抵达那个人口,这一次[人口]却不是处在某个整体的一个混乱表象之中,而是由许多规定和联系所构成的一个丰富总体。〇最初的途径,政治经济学在其起源上历史性地采取过了。比如17世纪的经济学家,总是开始于活的[lebendigen]整体,[即]人口、民族、国家、若干国家等等;他们却总是由此结束,即他们借由分析而找出少许确定的抽象的、一般的联系,像分工、货币、价值等等。一旦这些孤立的环节多多少少得到固定和概括,经济学的诸体系[die ökonomischen Systeme]就开始了,从那些简单的[环节],像劳动、分工、需求、交换价值而上升,直到国家、国际交换和世界市场。最后的[途径]\marginpar{248@701}显然是科学上正确的方法。〇具体者是具体的,因为它是许多规定之综合,从而是森罗万象[Mannigfaltigen,直译:多样性]之统一。在思维中,具体者因而作为综合之过程、作为结果、而非作为起点而呈现,虽然它是实际的起点从而也是直观和表象的起点。在最初的途径上,完满的表象被挥发为抽象的规定,在其次的[途径]上,抽象的诸规定导向具体者之再生产于思维之途径之上。〇黑格尔因此陷入幻想,把现实理解为那种思维之结果,那种思维自己内在结合、内在深化、由自己自身而自我运动,然而那种方法,从抽象者向具体者而上升,仅仅是就思维而言的方式,这种思维占有具体者、将之作为一个精神的具体者去再生产。但决不是具体者自身之形成过程。〇比如,最简单的经济学范畴,比如说交换价值,隶属[unterstellen]于人口,即在特定的诸关系中生产的人口;也隶属于家庭的、公社的或国家等的本质[wesen]之某些类型。交换价值决不能存在,除非作为某个已然既定的具体的、活的[lebendig]整体之抽象的、\textbf{片面的}联系。作为范畴,交换价值反而导向某种前[诺亚]大洪水期的定在。〇因此对于意识来说,而且哲学意识就是被如此规定,即现实的人之理解着的思维\footnote{或解作“理解着的思维是现实的人”。在德文版中,阳性名词Mensch的定冠词是第一格der表主语,而不是第四格den表宾语,或第二格des表所有格。费解,此处疑有误。由黑格尔哲学并未把思维当作人来看,或不应解作宾语,而应解作所有格。但解作宾语,似乎也能解得通,那就成了马克思自己对黑格尔思路的引伸。},从而被理解了的如此这般的世界,才是现实的[Wirkliche],因此诸范畴的运动表现为实际的生产行为,遗憾之处仅在于受到来自外界的一个激发,这一行为的结果就是世界;而在一定意义上正确的是——然而又是一个同义反复:具体的总体之作为思维总体,作为一个思维具体,事实上是思维、理解的一种产物;但决不是外在于或高出于直观和表象而思维并自我自身分娩的概念[Begriffs]之产物,而是从直观和表象到概念的加工之产物。〇整体,当它在头脑中显现为思维整体之时,是思维着的头脑的一个产物。这个头脑以唯独它才能做到的方式占有世界,这种\marginpar{249}方式不同于对这一世界的艺术的、宗教的、实践-精神的[praktisch-geistigen]占有。现实的主体一如既往地存续[bleiben]于头脑之外,坚持[bestehen]于其独立性之中;只要换句话说头脑仅仅思辨地、仅仅理论地对待\marginpar{@702}自己。因此同样对理论方法来说,主体、社会也必须作为表象的一贯前提而浮现出来。

〇但是,这些简单范畴是否也有一种独立的历史的或自然的实存[而又]先于具体呢?酌情而定。比如,黑格尔正确地以占有开始法哲学,[把占有]当作主体最简单的法的关系[rechtlichen Beziehung]。但是在家庭或主奴关系这些具体得多的关系之前,占有并不存在。恰好相反,若这样说则是正确的,即存在着[这样的]家庭和部落整体,它们还只占有而没有所有制。因此,相对于所有制,较简单范畴表现为较简单的家庭联合体或部落联合体的关系。在更高级的社会中,较简单范畴呈现为一个发达组织的较简单关系。更具体的基底——它的关系就是占有——却总已假定了。可以设想某个单独的野人拥有财产。那么占有却不是法律关系[Rechtsverhältnis]。占有历史地向家庭发展,这种说法并不正确。占有毋宁总是隶属于家庭这一“较具体的法律范畴”。此时无论如何就总保持[下面的情况]:简单范畴是[这样一些]关系的表现,在这些关系中不发达的具体可能已经实现,而更为多方面的联系或关系尚未成熟,它们在更具体的范畴中精神地表达出来;同时更发达的具体把同样的范畴保持为一种从属的关系。货币可以存在而且已经历史地实存,先于资本、银行、工资劳动等等而存在。依据这个方面就可以说,较简单的范畴可以表现一个较不\marginpar{250}发达的整体的支配性关系,或者一个较发达的整体的从属性关系,而较简单的范畴在历史上已经实存,先于整体之达到那发达方面,这种方面在一个更具体的范畴中表达出来。在这个限度内,抽象思维,即从最简单的[范畴]向结合体上升者,它的进程适应于实际的历史过程。

〇另一方面,可以说,存在着虽非常发达但历史上并不成熟的社会形式,其中经济之最高级形式,诸如协作、发达的分工等等,已经出现,[却]\marginpar{@703}不存在任何一枚货币,比如秘鲁。此外,在斯拉夫公社中,货币和它所依赖的交换,不是或很少是在单独的公社之内,而是在它们的边界上,在同其他公社的交往中出现的,正如完全错误的[做法]是把公社之中的交换确立为原始的组成元素。货币宁可说在开端上,相比于为了同一公社之内的成员们来说,更早地在不同公社相互间的联系中出现。进一步说:虽然货币很早很普遍地发挥了作用,[但是]它在古代之被指定为统治性要素,却仅是在片面的特定的民族即商业民族之中。而且甚至在最有教养的古代,在希腊人和罗马人中间,货币的充分发展,即在现代市民社会中作为前提的东西,仅仅出现在它们的衰落时代。因此这个非常简单的范畴,在历史上,只有当处在社会的最发达状态之中之时,才能展现其强度。[货币]决没有将所有的经济关系都跋涉而过。例如,在罗马帝国中,在它最繁荣昌盛之时,实物税和实物交付仍然是基础。货币本质[Geldwesen]其实仅在军队中才完全发达。它同样从未掌握劳动之整体。所以,虽然较简单的范畴在历史上可以已经存在而先于较具体的范畴,但是它之处在其充分的[向内深度和向外广度][注意intensiv和extensiv这两个词的前缀正好是相对的]的发展之中,恰恰[只]能够属于某个组合的社会形式,然而较具体的范畴在某个略较发达的社会形式中被比较充分地发展了。

〇\marginpar{251}劳动似乎是一个十分简单的范畴。劳动在其一般性上的表象——劳动一般——也是古老的。虽然“劳动”在经济学上在此种简单性上进行把握,但它同孕育这个简单抽象的诸种关系一样,是一个同样现代的范畴。〇例如,货币主义仍然完全客观地将财富规定为外化于货币中的东西。同这个观点相比,重工主义或重商主义将财富的源泉由物体转而规定为主体的活动,即商业劳动和工业劳动,就是一个巨大的进步。但仍然仅仅在赚钱这种狭隘性上对这种活动本身进行理解。同此种学说相对立的重\marginpar{@704}农主义,将劳动的一种特定形式——农业——创造性地规定为财富,甚至对象不再处于货币的伪装之中,而是作为产品一般,作为劳动的一般结果。这种产品还是与活动的狭隘性相适应,仍然作为自然规定的产品——农业产品,典型地\footnote{par excellence,法语。}是土地产品。

〇亚当·斯密所取得的巨大进展是,抛弃生产财富的活动的任一规定性——所得即劳动绝对,它既不是工业劳动,又不是商业劳动,也不是农业劳动,而既是这一个又是别一个。有了生产财富的活动的抽象一般性,那就有了作为规定财富的对象的产品一般的一般性,或者重又是劳动一般,然而是作为过去的、对象化的劳动。这个转变是多么艰难多么伟大,从亚当·斯密甚至还不时重又倒退回重农主义这一情况,就可以看出。〇那么这看起来也许就只是借此找到最简单最古老的联系的抽象表现,在这种存在于所有的社会形式之中的联系里,人们作为生产者出现。这从某种方面来看是正确的,从别种方面来看就不正确了。〇对于任一特殊种类的劳动的漠视,以诸种现实的劳动种类的一个非常发达的整体为前提,在这些劳动种类中不再有任何一种可以统治一切。因此最一般的抽象\marginpar{252}一般只是产生于最丰富的具体之发展,在这种发展中某种东西表现为许多东西所共有的,或一切东西所共通的。于是仅仅在单独的形式中进行考察最终就成为可能。〇另一方面,劳动一般这个抽象并不仅仅是具体的劳动整体的精神结果。对特殊的劳动的漠视适应于这种社会形式,其中个人简单地从一种劳动向另一种劳动转换,并且劳动的特定种类对于他们是偶然的,因而是无关紧要的。劳动在这里不仅在范畴上而且在现实中,都成为生产财富一般的手段,并且不再同个人相联系而处在某种残缺的特殊性中的规定。〇这种情况在市民社会的\marginpar{@705}最现代的定在形式美国是最发达的,因此在那里,“劳动”、“劳动一般”、劳动绝对(拉丁语)这个范畴的抽象,即现代经济学的出发点,才成为实际上真实的。因此,劳动这个最简单的抽象,虽然被现代经济学放到开端,而且表现出一种古老的并且对于所有社会形式都有效的联系,但是只能是作为最现代的社会的范畴,才在这种抽象中表现为实际上真实的。〇有人也许会说,在美国作为历史产物的东西,比如在俄罗斯人那里——对特殊的劳动的漠视——就表现为天然的素质。不过,是野蛮人具有素质,被动地承受一切,还是文明人自主地运用一切,存在非常大的区别。此外,对于俄罗斯人来说,对劳动的确定性的漠视,实际上同传统地固定在某种完全特殊的劳动之中相适应,他们只是由于外来的影响才被抛掷出来。

〇劳动的这一例子有力地证明了,正如甚至最抽象的诸范畴,不顾它们对所有时代的有效性——[这种有效性]恰好出于它们的抽象——就这些抽象本身的规定性来说,的确同样是历史性的诸关系的产物,并且它们的完全有效性仅仅是对于这些关系并在其之内才能具有。

〇\marginpar{253}市民社会是最发达、最多样的历史的生产组织。表现市民社会的诸关系的诸范畴,对其划分之理解,因而同时提供洞察于所有已灭亡的社会形态的划分和生产关系。借由这些已灭亡的社会形态的残片和元素,市民社会得以建立。在这些部分中,尚未克服的残余仍在市民社会之中残留[fortschleppen],单纯的征兆正向成熟的含义发展,等等。人体解剖对于猴体解剖是一个关键。较低等动物物种之中的较高等动物物种的征兆,反而只有当较高等动物物种自身已经知悉了之时,才能被理解。因此市民经济提供对古代[经济]等等的关键。〇但决不是按照经济学家们的方式:模糊所有历史差别而在所有的社会形态中都看到市民。只要认识了地租\marginpar{@706},就能理解贡金、什一税等等。但必须不把它们等同起来。〇此外,市民社会本身仅是一种矛盾的发展形态,所以,以前的诸形态的诸关系在其中被发现常常仅是完全萎缩的或完全歪曲的。例如公社所有制。因此,市民经济的诸范畴对于所有其他社会形态具有某种真实性,这个说法如果是真的,那也只能有保留地\footnote{cum grano salis,拉丁语}接受。这些范畴总能发达地、萎缩地或漫画化等等地包含上述关系于本质的区别之中。〇所谓的历史发展一般说来以此为根据:最后的形态将那些过去的形态视作朝向自己的诸阶段,而且,由于最后的形式极少并且仅在完全特定的条件下才有能力批判自己——这里所说的当然不是那种历史时代,其自身作为衰落时期显现出来者——[所以]总是片面地理解它们。基督教之能协助对以往神话的客观理解,仅当其自我批判在某种特定程度上或谓在可能性上\footnote{dunamei,希腊语}完成之时。因此,市民\marginpar{254}经济之得到对封建的、古典的和东方的经济的理解,仅当市民社会的自我批判开始之时。只要市民经济尚未神话地将自己同过去的经济纯然等同起来,那么市民经济对以往经济的批判,尤其对封建经济、市民经济还同之直接战斗过者的批判,就类似于基督教对异教、又或新教对天主教所施行的批判。

〇就像一般说来在每种历史的、社会的科学中那样,在经济范畴的进展中,总要牢牢把握:就如在现实中那样,在头脑中亦是如此,主体,此处乃现代市民社会,是既定的,并且诸范畴因而诸定在形式、诸实存规定,常常仅表现这一特定社会、这一主体之个别的诸方面,因此[现代市民社会]\textbf{同样在科学上}决非现在如此谈论它的时候才开始的。这一点要牢牢把握,因为它同样对划分\marginpar{@707}握有决定性。〇例如,没什么比从地租、地产来开始显得更为合乎自然,因为它捆绑于土地,即所有产品和所有定在的源头,以及农业,即所有相当巩固的社会之最早的生产形式。然而这错得无以复加。〇在每种社会形式中,都有某种特定的生产,它为所有其余的生产指定了等级和影响,因而其关系也为所有其余的关系指定了等级和影响。它是一种普遍的光线,在其中,所有其余的色彩都被渲染[getaucht,或译:浸入],并且由这光线以其特性加以修改。它是一种特殊的以太,确定了所有在其中显出的定在物之比重。〇例如就游牧民族来说。(纯粹的渔猎民族还处在真正的发展开始之时间点之外。)在它们中某种耕作形式零星出现。地产由此规定了。它是共有的,并且根据这些民族或多或少依然坚持其传统的情况,而或多或少地保留这一形式,例如斯拉夫人的公社所有制。在固定耕作的民族中(这种固定已然是重大的阶段)——在这种民族中这种[耕作]之占据统治地位就像在古典[社会]和封建[社会]中那样——甚至与它[耕作]适应的工\marginpar{255}业及其组织以及诸所有制形式,都具有或多或少地产的性质:要么完全依赖于它[耕作],就像在古代罗马人中那样,要么就像在中世纪那样,在城市及其关系中模仿着乡村的组织。甚至中世纪的资本——只要它并非纯粹的货币资本——作为传统的手工工具等等,也具有这些地产的性质。〇在市民社会中则相反,农业越来越成为一个纯粹的工业部门,并完全被资本所统治。地租亦然。在地产统治着的所有形态中,自然联系还占支配地位。在那些资本统治着的社会形式中,社会地历史地创造出来的因素占支配地位。没有资本,地租就无法被理解。没有地租,资本却完全能被理解。资本是市民社会的支配一切的经济力量,它必须构成起点和终点并放\marginpar{@708}在地产之前阐明。分别考察两者之后,它们的相互关系必须要被考察。

〇因此,既行不通也错误的是,以经济学范畴在历史上起决定性作用的次序,让它们相互跟随。宁可说,用来确定它们的排序的,是它们在现代市民社会中相互间具有的那种联系,而这刚好是下述东西的颠倒:作为其合乎自然的[次序]表现出来的东西,或者与历史发展之次序相符合的东西。它[即它们的排序]并不涉及那种关系,即在各种社会形式的相互序列中诸经济关系历史地占据的那种关系。更少[涉及]它们“在理念中”(\textbf{蒲鲁东})的排序(关于历史运动的一种模糊观念)。而是[涉及]它们在现代市民社会内部的划分[Gliederung]。

〇在古代世界的商业民族——腓尼基人和迦太基人——之中所显现的纯粹性(抽象的规定性),正是由于农业民族的统治性地位本身才出现的。\marginpar{256}在资本还不是社会的统治性的元素的地方,作为商业资本或货币资本的资本正是出现在这种抽象中。伦巴第人和犹太人对于从事农业的中世纪社会来说也占据这样的地位。

〇关于同样的范畴在不同的社会阶段上占有的不同地位,另一个例子就是:市民社会的最新形式之一,[股份公司][英文]。然而上述形式还在开端以享有特权并且配备了垄断权的大商业公司的形式出现过。

〇国民财富的概念——一种表象——在17世纪的经济学家那里是这样悄悄混入的:纯粹为了国家才创造财富,国家的力量确实与这种财富成比例。这个概念仍然部分地被18世纪的经济学家所延续。这还是一种无意识的伪善形式,在其中财富本身和财富的生产被宣布为现代国家的目的,而现代国家仅仅还被视作财富生产的手段。

〇划分显然要这样去做:1. 普遍的抽象的规定,因此或多\marginpar{@709}或少适应于所有社会形式,然而是在上面所阐明的含义上。2. 构成市民社会的内在划分并且作为基本阶级的根据的范畴。资本,工资劳动,土地所有制。它们相互间的联系。城市和乡村。三大社会阶级。上述阶级之间的交换。流通。信用活动(私人的)。3. 市民社会在国家形式上的总和。 在其与自身的联系之内考察。“非生产的”阶级。税。国债。国家信用。居民。侨民。移民。4. 国际的生产关系。国际的分工。国际的交换。进出口。汇率。5. 世界市场和危机。

\newpage

\subsection{4.生产。\\生产工具和生产关系。\\生产关系和交往关系。\\同生产关系、交往关系相关联的国家形式、意识形式。\\法律关系。\\家庭关系。}\marginpar{257}

便签,关于在这部分要提及而不应忘记的议题:

1. \textbf{战争}比和平更早胜任;方式,即怎样凭借战争并在军队中等等,特定的经济的关系诸如工资劳动和机器,比在市民社会内部更早发展。来自生产力与交往关系的关系在军队中尤其鲜明。

2. \textbf{迄今的观念的历史编纂同真实情况的关系。首当其冲的是所谓的文化史},老旧的宗教史和列国史。(与此同时也可以略谈一下迄今的历史编纂的各种方式。所谓客观的。主观的(伦理的等等)。哲学的。)

3. \textbf{第二三级之物},归根结底\textbf{派生的、转生的}而非原初的生产关系。在这里温习国际关系。

4. \marginpar{@710}\textbf{对这种见解之唯物主义性质的质疑。同自然主义的唯物主义的关系。}

5. \textbf{生产力(生产工具[生产介质])和生产关系这些概念的辩证法},\textbf{[这么]一种辩证法,}其边界应予规定且不取消真实的差别。

6. \textbf{物质生产之发展之不一致关系,例如同艺术[生产相比较]。}尤其,进步的概念不是在粗鄙的抽象上理解的。现代艺术之类。在实践的-社会的关系本身之内理解这种不均衡,则更为重要和困难。例如知识。合众国同欧洲的关系。这里要讨论的真正困难的要点却是,生产关系作为法律关系怎样走上不一致的\marginpar{258}发展。因此例如罗马私法(在刑法和公法中这种情况较少)同现代生产的关系。

7. \textbf{这一见解显现为必然的进展。然而是偶然之合理性。}怎样是。(也是自由之类)(交流工具的影响。世界历史并非向来就存在;历史之作为世界史,是结果。)

8. \textbf{起点自然起于自然规定};主观上和客观上。部族、民族之类。

就艺术而言,众所周知,其特定繁荣时期决不与一般的社会之发展成比例,因此也决不与似乎是社会机体的骨骼的物质基础成比例。例如同现代人形成对比的希腊人,或者莎士比亚也同样。艺术的某些形式例如史诗甚至确认,在其世界时代上作出的古典形态从未能被生产,只要艺术生产作为艺术生产进入;因此,在艺术的白令本身之内,艺术的某些显著的形态仅仅在艺术发展的某个不发展的阶段上才是可能的。如果在艺术的领域本身之内的不同艺术种类的关系上是这样的情况,那么不足为奇的是,在艺术的全部领域对一般的社会发展的关系上也是如此。困难仅仅在于这些矛盾的一般措辞。只要\marginpar{@711}它被详细说明,就一定会昭然若揭。

我们来处理例如希腊艺术、之后是莎士比亚同当代的关系。众所周知,希腊神话不仅仅是希腊艺术的宝库,还是其土壤。希腊的想象因而希腊的[神话]作为基础的、关于自然和社会关系的见解,能与走锭精纺机、铁路、机车和用电的电报机并存吗?在\marginpar{259}罗伯茨公司[Roberts et Co.]、避雷针和动产信用公司的面前,武尔坎[Vulkan]、丘比特[Jupiter]和赫尔墨斯[Hermes]该何去何从?所有神话都是在想象中并借想象来克服、掌握和刻画自然力:因此随着对自然力的真正统治而消亡。法玛[Fama]在印刷所广场之旁如何自容?希腊艺术以希腊神话为前提,这就是说,自然和社会形式自身已经借人民的想象在某种无意识的艺术方式中加工过了。这种神话就是希腊艺术的材料。并非任意一种的神话,也即并非任意一种对自然(此处包括所有具象物,因而也包含社会)的无意识的艺术加工。埃及神话决不会是希腊艺术的土壤和母胎。然而无论如何得有某种神话。因而决不是排除对自然的一切神话的、神话化的关系的社会发展;因此对艺术家要求某种独立于神话的想象。

从另一方面来看:阿基里斯能与火药和子弹并存吗?或者尤其《伊利亚特》能与活版印机甚或印刷机器并存吗?歌谣、传说、缪斯由于印刷机顽童不就必然绝迹,因而叙事诗的必要条件不就消亡了吗?

然而困难不在于理解希腊艺术和史诗同某种社会的发展形式相捆绑。困难是它们仍为我们提供艺术享受,并在某种程度上被视作准则和无从达到的榜样。

一个人不能再变成儿童,否则他就变得幼稚了。然而儿童的\marginpar{@712}天真烂漫难道不使他高兴,而他自己难道不应当再次在更高阶段上力求再生产这个天真烂漫的真理吗?在每个时代中,他所固有的特征难道不是以其自然真理性在儿童天性中复苏吗?为何历史上的人类童年,在它发展得最美丽的地方,\marginpar{260}不应该作为决不复返的阶段给予永恒魅力呢?粗鲁的、早熟的儿童也是有的。许多古代民族就属于这一类。标准的儿童则是希腊人。希腊人的艺术的魅力,对我们来说,并不与它在其上生长的那不发达的社会阶段相矛盾,而宁可说是后者的结果,并且宁可说不可分割地关联于此种情况:未成熟的社会条件,即希腊人的艺术能且只能兴起于其中者,决不能复返。

\end{document}
